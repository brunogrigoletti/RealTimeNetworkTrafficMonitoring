\documentclass[12pt]{article}
\usepackage[a4paper, total={6in, 9in}]{geometry}
\usepackage[utf8]{inputenc}
\usepackage[T1]{fontenc}
\usepackage{lmodern}
\usepackage{graphicx}
\usepackage{float}
\usepackage{csquotes}
\usepackage{lipsum}
\usepackage{enumitem}
\usepackage{xcolor}
\usepackage{listings}
\usepackage{hyperref}
\usepackage{listings}
\usepackage{tcolorbox}
\usepackage{seqsplit}

\renewcommand{\figurename}{Figura}
\hypersetup{colorlinks=true, urlcolor=blue}

\tcbuselibrary{listingsutf8}
\newtcblisting{terminal}{
    listing only,
    listing options={
        basicstyle=\ttfamily\color{white}\small,
        backgroundcolor=\color{black},
        keywordstyle=\color{cyan},
        commentstyle=\color{green},
        identifierstyle=\color{white},
        language=bash,
        breaklines=true,
        showstringspaces=false,
        numbers=none,
        deletekeywords={do},
    },
    colframe=black,
    colback=black,
    sharp corners,
    boxsep=1mm,
    left=2mm,
    right=2mm,
    top=1mm,
    bottom=1mm,
}

\lstdefinestyle{vscode}{
    language=Python,
    extendedchars=true,
    inputencoding=ansinew,
    tabsize=4,
    showstringspaces=false,
    numbers=left,
    firstnumber=1,
    commentstyle=\color{gray},        
    keywordstyle=\color{blue},        
    stringstyle=\color{orange},
    rulecolor=\color{black},
    basicstyle=\small\ttfamily,
    breaklines=true,
    numberstyle=\tiny,
    aboveskip=1em,
    belowskip=1em,
}

\graphicspath{{./images/}}

\title{\textbf{Monitor de Tráfego de Rede em Tempo Real}}
\date{Laboratório de Redes de Computadores}
\author{Bruno Laitano e Bianca Alves}

\begin{document}

\maketitle

\section{Introdução}
O presente relatório detalha a implementação de uma ferramenta de monitoramento de tráfego de rede em tempo real. O sistema desenvolvido é capaz de capturar, interpretar e classificar pacotes de rede, além de fornecer uma interface de usuário simples para que seja possível visualizar as estatísticas do tráfego. Além disso, cada captura incrementa arquivos \texttt{.xlsx}, os quais mantêm registrado um histórico de cada tipo de pacote recebido.

\quad Os logs da camada de Enlace (\texttt{layer2.xlsx}) registram a data e a hora em que o quadro foi capturado; o endereço MAC de origem e o endereço MAC de destino; o protocolo que o \emph{frame} carrega no formato hexadecimal; e, por fim, o tamanho total do quadro em \emph{bytes}. No caso da camada de Rede (\texttt{layer3.xlsx}), registra-se a data e a hora em que o pacote foi capturado; o nome do protocolo (IPv4, IPv6 ou ARP); o endereço IP de origem e o endereço IP de destino; o identificador do protocolo carregado no pacote; e, finalmente, o tamanho total do pacote em \emph{bytes}. Ainda, no caso da camada de Transporte (\texttt{layer4.xlsx}), são registrados a data e a hora em que o pacote foi capturado; o nome do protocolo (TCP, UDP, ICMP ou ICMPv6); o endereço IP de origem; a porta de origem; o endereço IP de destino; a porta de destino; e, enfim, o tamanho total do pacote em \emph{bytes}.

\quad Finalmente, a escolha pelo uso do pacote \texttt{winpcapy} neste projeto se deve à sua capacidade de fornecer acesso direto e eficiente à captura de pacotes de rede em sistemas \emph{Windows}, aproveitando a biblioteca \emph{WinPcap}. O \texttt{winpcapy} permite a interceptação de pacotes em baixo nível, possibilitando a análise detalhada dos protocolos das camadas 2, 3 e 4 da pilha TCP/IP, o que é fundamental para a nossa aplicação de monitoramento de tráfego de rede em tempo real.

\section{Extração}
A identificação do cabeçalho de cada tipo de pacote ou quadro é realizada com base na operação \texttt{unpack}, oferecida pelo pacote \texttt{struct}. Este módulo é utilizado para converter uma sequência de \emph{bytes} em campos de dados estruturados, conforme o formato especificado. \emph{Strings} de formato compacto descrevem as conversões pretendidas de/para valores Python.

\subsection{\emph{Ethernet}}
Inicialmente, é preciso extrair e processar o cabeçalho \emph{Ethernet} do pacote capturado. Isso é realizado identificando-se os primeiros 14 bytes do pacote.

\begin{lstlisting}[style=vscode]
eth_header = pkt_data[:14]
eth = struct.unpack('!6s6sH', eth_header)
\end{lstlisting}

\quad A \emph{String} \texttt{!6s6sH} referencia os seguintes valores neste universo de 14 \emph{bytes}:

\begin{itemize}
    \item \texttt{!}: ordem de \emph{bytes} de rede;
    \item \texttt{6s}: 6 \emph{bytes} para o endereço MAC de destino;
    \item \texttt{6s}: 6 \emph{bytes} para o endereço MAC de origem;
    \item \texttt{H}: 2 \emph{bytes} para o campo EtherType (identifica o protocolo da camada superior).
\end{itemize}

\subsection{IPv4}
Na sequência, extraímos e processamos o cabeçalho IPv4, correspondido entre os \emph{bytes} 14 e 34 do pacote. O cabeçalho IPv4 padrão possui 20 \emph{bytes} e inicia logo após o cabeçalho \emph{Ethernet}.

\begin{lstlisting}[style=vscode]
ip4_header = pkt_data[14:34]
ipv4 = struct.unpack('!BBHHHBBH4s4s', ip4_header)
\end{lstlisting}

\quad A \emph{String} \texttt{!BBHHHBBH4s4s} referencia os seguintes valores neste universo de 20 \emph{bytes}:

\begin{itemize}
    \item \texttt{!}: ordem de \emph{bytes} de rede;
    \item \texttt{B}: 1 \emph{byte} para Versão + IHL (\emph{Internet Header Length});
    \item \texttt{B}: 1 \emph{byte} para \emph{Type of Service} (ToS);
    \item \texttt{H}: 2 \emph{bytes} para \emph{Total Length} do pacote;
    \item \texttt{H}: 2 \emph{bytes} para \emph{Identification};
    \item \texttt{H}: 2 \emph{bytes} para \emph{Flags} + \emph{Fragment Offset};
    \item \texttt{B}: 1 \emph{byte} para \emph{Time To Live} (TTL);
    \item \texttt{B}: 1 \emph{byte} para \emph{Protocol} (o identificador do protocolo de camada superior);
    \item \texttt{H}: 2 \emph{bytes} para \emph{Header Checksum};
    \item \texttt{4s}: 4 \emph{bytes} para o IP de origem;
    \item \texttt{4s}: 4 \emph{bytes} para o IP de destino.
\end{itemize}

\subsection{IPv6}
Ainda no contexto da camada de Rede, extraímos e processamos o cabeçalho IPv6, correspondido entre os \emph{bytes} 14 e 54 do pacote. O cabeçalho IPv6 possui sempre 40 \emph{bytes} e também começa logo após o cabeçalho \emph{Ethernet}.

\begin{lstlisting}[style=vscode]
ip6_header = pkt_data[14:54]
ipv6 = struct.unpack('!IHBB16s16s', ip6_header)
\end{lstlisting}

\quad A \emph{String} \texttt{!IHBB16s16s} referencia os seguintes valores neste universo de 40 \emph{bytes}:

\begin{itemize}
    \item \texttt{!}: ordem de \emph{bytes} de rede;
    \item \texttt{I}: 4 \emph{bytes} para Versão, \emph{Traffic Class} e \emph{Flow Label};
    \item \texttt{H}: 2 \emph{bytes} para \emph{Payload Length};
    \item \texttt{B}: 1 \emph{byte} para \emph{Next Header} (identificador do protocolo de transporte);
    \item \texttt{B}: 1 \emph{byte} para \emph{Hop Limit} (TTL do IPv6);
    \item \texttt{16s}: 16 \emph{bytes} para o IP de origem;
    \item \texttt{16s}: 16 \emph{bytes} para o IP de destino.
\end{itemize}

\subsection{ARP}
Para a identificação de pacotes ARP, extraímos e processamos o cabeçalho correspondente entre os \emph{bytes} 14 e 42 do pacote. O cabeçalho ARP possui 28 \emph{bytes} e também começa logo após o cabeçalho \emph{Ethernet}.

\begin{lstlisting}[style=vscode]
arp_header = pkt_data[14:42]
arp = struct.unpack('!HHBBH6s4s6s4s', arp_header)
\end{lstlisting}

\quad A \emph{String} \texttt{!HHBBH6s4s6s4s} referencia os seguintes valores neste universo de 28 \emph{bytes}:

\begin{itemize}
    \item \texttt{!}: ordem de \emph{bytes} de rede;
    \item \texttt{H}: 2 \emph{bytes} para o \emph{Hardware Type};
    \item \texttt{H}: 2 \emph{bytes} para o \emph{Protocol Type};
    \item \texttt{B}: 1 \emph{byte} para o \emph{Hardware Size};
    \item \texttt{B}: 1 \emph{byte} para o \emph{Protocol Size};
    \item \texttt{H}: 2 \emph{bytes} para o \emph{Opcode};
    \item \texttt{6s}: 6 \emph{bytes} para o endereço MAC de origem;
    \item \texttt{4s}: 4 \emph{bytes} para o endereço IP de origem;
    \item \texttt{6s}: 6 \emph{bytes} para o endereço MAC de destino;
    \item \texttt{4s}: 4 \emph{bytes} para o endereço IP de destino.
\end{itemize}

\subsection{TCP e UDP}
Já no contexto da camada de Transporte, extraímos e processamos os cabeçalhos dos protocolos TCP e UDP. No caso do \emph{Transmission Control Protocol}, o cabeçalho é reconhecido entre os \emph{bytes} 34 e 54 do pacote. O cabeçalho TCP padrão possui 20 \emph{bytes} e inicia logo após o cabeçalho IPv4 (que começa no \emph{byte} 14).

\quad Por outro lado, no caso do \emph{User Datagram Protocol}, o cabeçalho é reconhecido entre os \emph{bytes} 34 e 42 do pacote. O cabeçalho UDP tem 8 \emph{bytes} e, neste caso, começa após o cabeçalho IPv6.

\begin{lstlisting}[style=vscode]
transport_header = None
    if ipv4[6] == 6:
        transport_header = pkt_data[34:54]
        tcp = struct.unpack('!HHLLBBHHH', transport_header)
        ...
    elif ipv6[2] == 17:
        transport_header = pkt_data[34:42]
        udp = struct.unpack('!HHHH', transport_header)
\end{lstlisting}

\quad No caso do TCP, a \emph{String} \texttt{!HHLLBBHHH} referencia os seguintes valores neste universo de 20 \emph{bytes}:

\begin{itemize}
    \item \texttt{!}: ordem de \emph{bytes} de rede;
    \item \texttt{H}: 2 \emph{bytes} para Porta de Origem;
    \item \texttt{H}: 2 \emph{bytes} para Porta de Destino;
    \item \texttt{L}: 4 \emph{bytes} para Número de Sequência;
    \item \texttt{L}: 4 \emph{bytes} para Número de \emph{Acknowledgment};
    \item \texttt{B}: 1 \emph{byte} para \emph{Data Offset} + \emph{Reserved} + \emph{Flags};
    \item \texttt{B}: 1 \emph{byte} para \emph{Window Size};
    \item \texttt{H}: 2 \emph{bytes} para \emph{Checksum};
    \item \texttt{H}: 2 \emph{bytes} para \emph{Urgent Pointer};
    \item \texttt{H}: 2 \emph{bytes} para Opções (se houver).
\end{itemize}

\quad No caso do UDP, a \emph{String} \texttt{!HHHH} referencia os seguintes valores neste universo de 8 \emph{bytes}:

\begin{itemize}
    \item \texttt{!}: ordem de \emph{bytes} de rede;
    \item \texttt{H}: 2 \emph{bytes} para Porta de Origem;
    \item \texttt{H}: 2 \emph{bytes} para Porta de Destino;
    \item \texttt{H}: 2 \emph{bytes} para Comprimento;
    \item \texttt{H}: 2 \emph{bytes} para \emph{Checksum}.
\end{itemize}

\subsection{ICMP e ICMPv6}
Ainda na camada de Transporte, extraímos e processamos os cabeçalhos dos protocolos ICMP e ICMPv6. No caso do \emph{Internet Control Message Protocol}, o cabeçalho é reconhecido entre os \emph{bytes} 34 e 42 do pacote. O cabeçalho ICMP padrão possui 8 \emph{bytes} e inicia logo após o cabeçalho IPv4.

\quad Por outro lado, no caso do \emph{Internet Control Message Protocol Version 6}, o cabeçalho é reconhecido entre os \emph{bytes} 54 e 62. O cabeçalho ICMPv6 padrão possui 8 \emph{bytes} e inicia logo após o cabeçalho IPv6.

\begin{lstlisting}[style=vscode]
...
    elif ipv4[6] == 1:
        transport_header = pkt_data[34:42]
        icmp = struct.unpack('!BBHHH', transport_header)
        ...
    elif ipv6[2] == 58:
        transport_header = pkt_data[54:62]
        icmpv6 = struct.unpack('!BBHHH', transport_header)
\end{lstlisting}

\quad Em ambos os cenários, a \emph{String} \texttt{!BBHHH} referencia os seguintes valores neste universo de 8 \emph{bytes}:

\begin{itemize}
    \item \texttt{!}: ordem de \emph{bytes} de rede;
    \item \texttt{B}: 1 \emph{byte} para o Tipo;
    \item \texttt{B}: 1 \emph{byte} para o Código;
    \item \texttt{H}: 2 \emph{bytes} para o \emph{Checksum};
    \item \texttt{H}: 2 \emph{bytes} para o Identificador (usado em \emph{Echo Request}/\emph{Reply});
    \item \texttt{H}: 2 \emph{bytes} para o Número de Sequência (também usado em \emph{Echo Request}/\emph{Reply}).
\end{itemize}

\section{Interface de usuário}
A interação do usuário com o sistema de captura de tráfego de rede desenvolvido é baseada em dois comandos: \texttt{cat} e \texttt{stop}. Ao executar o algoritmo, uma lista de interfaces de rede disponíveis é exibida, permitindo ao usuário escolher em qual das interfaces em sua estação a captura deve ser efetivamente realizada.

\begin{lstlisting}[style=vscode]
def list_interfaces():
    print("\nAvailable interfaces:")
    with WinPcapDevices() as devices:
        device_list = []
        for i, device in enumerate(devices):
            desc = device.description.decode() if isinstance(device.description, bytes) else device.description
            desc = desc if desc else "No description available"
            print(f"{i}: {desc}")
            device_list.append(desc)

        idx = int(input("\nType the interface number to start capturing: "))
        return device_list[idx]

stop_event = threading.Event()

def capture_packets(interface):
    try:
        WinPcapUtils.capture_on(interface, packet_callback)
    except Exception as e:
        print(f"Capture error: {e}")
\end{lstlisting}

\quad Escolhida a interface, a captura será iniciada após o comando \texttt{cat}. Nesse momento, os arquivos \texttt{.xlsx} são criados (caso ainda não existam) e são alimentados com os dados já mencionados a respeito de cada um dos pacotes que o sistema é capaz de identificar e monitorar.

\section{Registro dos arquivos}
A escolha do pacote \texttt{openpyxl} para a criação de arquivos no formato \texttt{.xlsx} se deve à sua capacidade de criar, ler e modificar planilhas de maneira eficiente. Assim, conforme a captura é realizada em tempo real, as informações pré-determinadas a respeito de cada pacote são registradas e organizadas nas tabelas, possibilitando a posterior análise dos resultados em ferramentas como o \emph{Microsoft Excel}.

\quad O código realiza uma checagem prévia para identificar a existência de arquivos de log antigos antes de iniciar uma nova captura de pacotes. Nesse sentido, a função \texttt{delete\_existing\_logs()} verifica se cada arquivo esperado (para as camadas 2, 3 e 4 da pilha TCP/IP) está presente no diretório. Caso o arquivo exista, ele é removido de forma a garantir que os dados registrados em execuções anteriores não interfiram nos resultados da nova captura.

\begin{lstlisting}[style=vscode]
excel_lock = threading.Lock()

is_shutting_down = False

LAYER2_LOG = "layer2.xlsx"
LAYER3_LOG = "layer3.xlsx"
LAYER4_LOG = "layer4.xlsx"

def delete_existing_logs():
    log_files = [LAYER2_LOG, LAYER3_LOG, LAYER4_LOG]
    print("Checking for existing log files...")
    for file in log_files:
        try:
            if os.path.exists(file):
                os.remove(file)
                print(f"Deleted existing log file: {file}")
        except Exception as e:
            print(f"Error deleting {file}: {e}")

def write_excel_log(filename, header, row):
    with excel_lock:
        file_exists = os.path.exists(filename) and os.path.getsize(filename) > 0
        try:
            if file_exists:
                wb = load_workbook(filename)
                ws = wb.active
            else:
                wb = Workbook()
                ws = wb.active
                ws.append(header)
            ws.append(row)
            wb.save(filename)
        except Exception as e:
            print(f"Error on file '{filename}': {e}")
\end{lstlisting}

\section{Estatísticas de captura}
Além do comando \texttt{cat}, que inicia a captura, o usuário pode interrompê-la a qualquer momento utilizando o comando \texttt{stop}. No momento em que a captura é interrompida, são impressas na tela as estatísticas relacionadas a cada pacote capturado, em especial a quantidade de quadros e pacotes identificados durante o monitoramento do tráfego de rede na interface indicada.

\quad Abaixo, você poderá ver um exemplo dos resultados impressos no terminal após o encerramento da captura por parte do usuário:

\begin{terminal}
Capture stopped!
==================================================
Summary:
Ethernet frames: 203
IPv4 packets:    196
IPv6 packets:    2
ARP packets:     1
TCP segments:    174
UDP datagrams:   2
ICMP messages:   8
ICMPv6 messages: 0
==================================================
Shutdown complete. Excel files saved safely.
\end{terminal}

\end{document}